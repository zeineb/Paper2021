\documentclass[10pt]{article}
\usepackage[usenames]{color} %used for font color
\usepackage{amssymb} %maths
\usepackage{amsmath} %maths
\usepackage[utf8]{inputenc} %useful to type directly diacritic characters
\begin{document}
\section{Introduction}
Secure software can be obtained out of two distinct processes:
Security by Design, and Security by Certification.
The former approach has been quite extensively formalized as it builds upon models,, which are verified to ensure security properties are attained from and from which software is then derived manually or automatically. In contrast, the latter approach has been quite always informal in both specifying security best practices ans verifying that the code produced conforms to them.
Building secure software is much more difficult than finding vulnerabilities and weaknesses
in a program. It requires establishing the mechanisms to protect the software and avoid flaws
that can be the source of serious vulnerabilities. Secure software can be obtained out of two
distinct processes: security by design, and security by certification. The former approach
has been quite extensively formalized as it builds upon models, which are verified to ensure

security properties are attained and from which software is then derived manually or auto-
matically. In contrast, the latter approach has always been quite informal in both specifying

security best practices and verifying that the code produced conforms to them. The former
technique, also called model-driven engineering for security, has been extensively developed in
the academic community, and follows a specify, model, verify, and then implement approach.
It is well suited to classical software development featuring for instance an organized V-cycle,
and security objectives generally correspond to security properties expected from software

components or communication protocols (see for instance [155]). In contrast, the latter tech-
nique follows a build, then detect flaws approach, based on the description of security best

practices and programming style and idioms. Apart from the safety-critical domain, this

approach has received a much wider acceptance in the industry because security is often ad-
dressed from a developer point of view, in a programmatic style, rather than from a security

architect perspective, as through a comprehensive model. A number of security guidelines is
defined forming a program-oriented security policy that the developer has to adhere to. This
model is also adapted to very different styles of software engineering including agile methods.
Unfortunately, this technique lacks automation and formality in the way best practices are
specified and verified.
In our work, we focus on the latter approach and describe how security guidelines might
be captured by security experts and verified formally by developers. Our technique relies
on abstracting actions in a program based on modularity, and on combining model checking
together with control and information flow analysis. Our goal is to formalize the existing
body of knowledge in security best practices using formulas in the MCL language and to
conduct formal verification of the conformance of programs with such security guidelines.
\section{Security guidelines}
\subsection {Formal specification}
Let us consider this guideline:
<em> It is good practice to be responsible for freeing all resources you allocate and to be consistent with how and where you free memory in a function. If you allocate memory that you intend to free upon completion of the function, you must be sure to free the memory at all exit points for that function including error conditions.
 </em> \cite{https://cwe.mitre.org/data/definitions/404.html}
It mentions that the developer needs to free all used resources with the objective of releasing resources properly.
If we want to formalize it in MCL:
use(resource) -> free(resource)
This formalization explains that for each use of resource, it should be followed by a release of it.

\subsection {From Security Requirements to Verifiable Properties}
\subsection {Formal verification}
\section{Formal specification}
\subsection {The Model Checking Language}
\subsection {Construction of the formal specification of security guidelines in MCL}
\section{Program model Construction}
\subsection {Verification Framework}
\subsection {Related Work}
\section{Conclusion}
\end{document}
