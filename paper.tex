\documentclass[10pt]{article}
\usepackage[usenames]{color} %used for font color
\usepackage{amssymb} %maths
\usepackage{amsmath} %maths
\usepackage[utf8]{inputenc} %useful to type directly diacritic characters
\begin{document}
\section{Introduction}
Secure software can be obtained out of two distinct processes:
Security by Design, and Security by Certification.
The former approach has been quite extensively formalized as it builds upon models,, which are verified to ensure security properties are attained from and from which software is then derived manually or automatically. In contrast, the latter approach has been quite always informal in both specifying security best practices ans verifying that the code produced conforms to them.
Building secure software is much more difficult than finding vulnerabilities and weaknesses
in a program. It requires establishing the mechanisms to protect the software and avoid flaws
that can be the source of serious vulnerabilities. Secure software can be obtained out of two
distinct processes: security by design, and security by certification. The former approach
has been quite extensively formalized as it builds upon models, which are verified to ensure

security properties are attained and from which software is then derived manually or auto-
matically. In contrast, the latter approach has always been quite informal in both specifying

security best practices and verifying that the code produced conforms to them. The former
technique, also called model-driven engineering for security, has been extensively developed in
the academic community, and follows a specify, model, verify, and then implement approach.
It is well suited to classical software development featuring for instance an organized V-cycle,
and security objectives generally correspond to security properties expected from software

components or communication protocols (see for instance [155]). In contrast, the latter tech-
nique follows a build, then detect flaws approach, based on the description of security best

practices and programming style and idioms. Apart from the safety-critical domain, this

approach has received a much wider acceptance in the industry because security is often ad-
dressed from a developer point of view, in a programmatic style, rather than from a security

architect perspective, as through a comprehensive model. A number of security guidelines is
defined forming a program-oriented security policy that the developer has to adhere to. This
model is also adapted to very different styles of software engineering including agile methods.
Unfortunately, this technique lacks automation and formality in the way best practices are
specified and verified.
In our work, we focus on the latter approach and describe how security guidelines might
be captured by security experts and verified formally by developers. Our technique relies
on abstracting actions in a program based on modularity, and on combining model checking
together with control and information flow analysis. Our goal is to formalize the existing
body of knowledge in security best practices using formulas in the MCL language and to
conduct formal verification of the conformance of programs with such security guidelines.
\section{Security guidelines}
\subsection {Formal specification}
Let us consider this guideline:
<em> It is good practice to be responsible for freeing all resources you allocate and to be consistent with how and where you free memory in a function. If you allocate memory that you intend to free upon completion of the function, you must be sure to free the memory at all exit points for that function including error conditions.
 </em> \cite{https://cwe.mitre.org/data/definitions/404.html}
It mentions that the developer needs to free all used resources with the objective of releasing resources properly.
If we want to formalize it in MCL:
use(resource) -> free(resource)
This formalization explains that for each use of resource, it should be followed by a release of it.

\subsection {From Security Requirements to Verifiable Properties}
Transforming the informal description of security guidelines into exploitable mathemat-
ical formulas is an important part in this thesis work. We stressed in Chapter 2 the

main issues that we have identified with respect to security guidelines, and the difficul-
ties that can occur when they are interpreted and implemented by software developers,

and verified afterwards by testers or code auditors. In addition, security guideline pre-
sentations differ from one source to another, hence, there is a need for a centralized and

uniform approach to represent them. In this Chapter, we present the methodology that
we provide to the security expert in order to help him build the guideline formulas. We focus on the extensibility as well as on the automation that we can bring to this phase

to strip away the difficulties which the security expert can face when building the guide-
lines formulas, especially those that involve dependent data flows. We explain in details

the flow of operations, and pinpoint the effort we have undertaken to automate the
formalization of security guidelines. For a clear understanding of the specification flow,
we consider concrete examples of guidelines that Brian, the security expert, formalizes,
starting from their informal description.
The main idea behind this Chapter is to transform the security guidelines written in

natural language, into exploitable mathematical formulas. The advantage of this for-
malization is to strip away the ambiguities of natural language, and to increase the

precision. This phase produces a formal specification of the security guidelines.

Since the guidelines that we want to specify reflect the desired program behavior, we
can conclude that what is needed is a formalism that allows to describe the execution
of the program. Those guidelines are generally abstracted away in formal models as
verification problems are undecidable for infinite systems.
Adopting Temporal Logics [105] seems a wise direction as they are suited to specify with
rigorous semantics the desired properties in the form of sequences of atomic propositions

or statements representing the behavior of the system. In the literature, different tem-
poral logics have been studied and proposed for the specification of program properties.

"Temporal" refers to the sequencing and chronological order of states, and not to the
notion of physical time.
Different Temporal Logic formalisms exist and are widely adopted, such as LTL and
CTL, with their different variants. The syntax of linear temporal logic (LTL) [125] is
defined inductively using the standard boolean operators, and the temporal operators
X(next), F (eventually), G (always OR globally), and U (strong until) by the following
abstract syntax equation:

φ := p | ¬p | p ∧ q | Xp | Gp | Fp | p U q

\subsection{Model Checking Language}
In this section, we explain in more details the Model Checking Language (MCL) that
we adopted for the formal specification of security guidelines. The formalism that we
have used is very intuitive in the sense that buidling formulas is close to the natural
processing of texts. We adopt the MCL logic [103] that addresses this crucial matter of
representing and handling data. It allows reasoning naturally about systems described
in value-passing process algebras such as LOTOS.

MCL (Model Checking Language) is an extension of the alternation-free regular μ-
calculus with facilities for manipulating data in a manner consistent with their usage

in the system definition. The MCL formulas are logical formulas built over regular
expressions using boolean operators, modality operators (the necessity operator denoted
by ([ ]) and the possibility operator denoted by (h i) and the maximal fixed point
operator (denoted by μ).
By using data, one can specify state machines with an infinite action alphabet, in
which quantification over data can be used, and propositions and variables may have
data parameters. MCL is a powerful language that combines value passing mechanisms
together with regular expressions, and constructs similar to programming languages
Most of the guidelines we studied can be expressed as usual safety properties that we
encode in the MCL formalism by [φ]false formula, stating the absence of bad execution
sequences characterized by regular formulas φ or as basic liveness properties, encoded
by hφitrue and stating the existence of good execution sequences characterized by φ.
Regular formulas φ are built over action formulas and the standard regular expression

operators namely concatenation (.), choice (|), and transitive-reflexive closure (∗). Ac-
tion formulas are built over action patterns and boolean connectors. Action patterns are

of two different kinds: action patterns for matching values denoted by ({A !e1 . . .!en}),
or action patterns for extracting and storing values denoted by ({A ?x1 :T1 . . .?xn :Tn}),
where A is the action name, ei are expressions (data variables or functions), xi are data
variables and Ti are types, namely Int, Bool, String. The true constant is used to
match a value of any action formula.
Syntax of the Model Checking Language Now if we go back to the property "No
writes on a file after it is closed" and formalize it with MCL. The parameterized action
patterns that we define and use to build our formula are:

– {File ?file : String} designates the action pattern for storing the file path "file"
of type String
– {Write !file} designates the action pattern of writing on the file file
– {Close !file} designates the action pattern of closing the file file

[true*.{File ?file:String}.(not({Close !msg}))*.{Write !msg}] false

This formula specifies that if the file file is not closed, and that the action of writing on
this file occurred, then this formula is false. This means that if we capture this behavior
in our program, then a violation should be detected. The "(true*)" indicates any action.
The formula can be read as follows: after any action occurring, if there exists a File file
(File ?file:String), and that this file is not closed (not(Close !msg)), followed at a some
point in time by a write on this file (Write !msg), then a violation should be raised.
Note that this MCL formula is built upon regular expressions operators (., *), and note
also that they are familiar to software developers.

\subsection{Construction of the Formal Specification of Security
Guidelines in Model Checking Language}
Having presented the formalism that we adopt to specify the security guidelines, we
examine its applicability on a number of guidelines that we have considered. Before
proceeding to the formal specification, we explain the methodology that we provide
to the security expert to build the MCL mathematical formulas from the guidelines
descriptions.
The formalization flow is the following:

– Extraction of the keywords (action patterns) and the key-concepts from the guide-
line textual description

– Construction of the basic MCL formula
– Construction of the MCL formula involving explicit and implicit dependencies
Each step as well as the transition from one step to another are depicted in detail in
the next sections.
The proposed formalization in the MCL language presents the key concepts required by
each of the guidelines. The challenge is to define those concepts and actions and have
a unified yet extensible set of keywords that can be used by the security expert(s) to
formalize the guidelines. We tried to reduce the expert intervention by introducing the
Security Knowledge Base (See Section 7.5) which is a central repository containing the
different labels used for composing the guidelines, together with their description and
semantic meaning.

\subsubsection{Extracting the Key-concepts}
First, Brian, the security expert, is faced with the guideline written in natural lan-
guage, and needs to identify the key elements at the heart of the guideline (sensitive

data, security mechanism, etc.) before proceeding to the specification. Brian refers
to the Security Knowledge Base (SKB) (see Section 7.5), and more specifically to the
security guideline patterns and the labels that are used to build the formulas; they
can be used to guide Brian when building the formulas based on the existing patterns.
This operation will result in a basic formula; the term basic means that the formula
involves the most intuitive and simplistic pattern that considers only the data assets

to be protected. The basic formula does not involve the explicit or the implicit de-
pendencies that data might have. For instance, if the data was assigned to a variable (implicitly or explicitly), this data dependency will not be captured in the basic formula.
Let us take a closer look at some of the guidelines that we have surveyed, and explain
how Brian proceeds to build them. We have considered the three guidelines IDS03-J:Do
not log unsanitized user input [30], MSC62-J:Store passwords using a hash
function [40] and OWASP: Store unencrypted keys away from the encrypted
data [119]. These three guidelines illustrate several concepts, such as the notion of
single or multiple parameters, patterns, etc.
Let us now look at the keywords and the concepts that Brian can extract from the textual
description, and provide explanation for each of them. It is important to mention the
following; considering only the textual title or the name of the guidelines is not sufficient,
we need also to look at the detailed textual explanations that provide very useful details,
such as the cryptographic algorithms that should be used, or the hashing method to be
applied in a reliable manner. For example, in the guideline MSC62-J:Store passwords
using a hash function [40], the hashing should be implemented through a robust
and reliable algorithm, and those details appear within the detailed description of the

guideline. CERT provides the following: Avoid defective functions such as the Message-
Digest Algorithm (MD5). Hash functions such as Secure Hash Algorithm (SHA)-1 and

SHA-2 are maintained by the National Security Agency and are currently considered
safe. However, SHA-1 is no longer a reliable hashing algorithm, it got recently broken
1 2
From this perspective, if we consider that any password hashing operation as safe, even
using a non-reliable algorithm, then this might lead to a violation of the password’s
integrity. This brought further reflection on the dictionary enrichment considering a
multitude of elements such as: the hashing/encryption algorithm to use in what context,
the key length, etc. To give another example, the usage of DES for symmetric encryption
purposes in indicated in our SKB as prohibited, as well as the hashing through MD5.

\subsubsection{Building the Basic Formulas}
Once the key-concepts are retrieved from the guideline textual description, Brian might
refer to the Security Knowledge Base to check whether the extracted words have a
matching entry. He might for instance find synonyms or semantically equivalent words
(labels) that can use to build the formula. Having selected the key concepts, Brian
proceeds to building the more natural and intuitive MCL formula.
Since establishing a single dictionary is the key to building the formulas, we identified
the need to have a central repository containing the guidelines relevant details, such
as the patterns and the labels used to build the guidelines formulas. Brian can access
this central repository (Security Knowledge Base 7.5) and can use it as a reference model. When formalizing the guidelines, Brian extracts the key concepts from the
textual description, and at this level, he consults the dictionary to see if the identified
concepts have already been used in other guidelines patterns, and to check if there exists
a pattern that he can make use of to build the formula. If not, Brian introduces the
new labels, and attributes semantics reflecting the meaning of those new labels. We
need to stress that our central repository (Security Knowledge Base in section 7.5) can
be extended to cover a wide range of security concepts.

\subsubsection{Patterns for the Formal Specification of Security Guidelines}
When formalizing the security guidelines described in the previous section, we have
noticed that there are recurring patterns. For instance, the notion of precedence arises
in different guidelines, and constrains the order of actions.
Since we aim at leveraging the workload and bringing assistance and guidance to the se-
curity expert, we decided to further simplify the formalization of the security guidelines

and propose property patterns. Thanks to these, the security expert will hopefully not
need to have full knowledge about the formal language expressiveness nor to master its
idioms. He would eventually prefer to have more guidance on how to use the language
features. This would improve the transformation of the security guidelines from natural
language to formulas.
Similarly to Dywer’s work \cite{Dwyer::1999}, we propose a catalog of patterns for security guidelines.

These patterns identified in [103] will allow programmers who are not necessarily ex-
perts in formal language to read a formal specification. As proposed in [103], we encode

guidelines as a collection of property patterns intended to simplify the specification ac-
tivity. These patterns belong to particular classes: Absence, Existence, and Universality

and occur in various scopes: Before, After, and Between. Those classes reflect specific
requirements such as the absence of an action, precedence of actions, response of an
action that is triggered by another action, etc. The proposed patterns were adopted by
the Bandera Specification Language \cite{corbett2002expressing}.


\subsection {Formal verification}
\section{Formal specification}
\subsection {The Model Checking Language}
\subsection {Construction of the formal specification of security guidelines in MCL}
\section{Program model Construction}
\subsection {Verification Framework}
\subsection {Related Work}
\section{Conclusion}


%% The Bibliography
%% ==> You need a file 'literature.bib' for this.
%% ==> You need to run BibTeX for this (Project | Properties... | Uses BibTeX)
\addcontentsline{toc}{chapter}{Bibliography} %'Bibliography' into toc
%\nocite{*} %Even non-cited BibTeX-Entries will be shown.
\bibliographystyle{plain} %Style of Bibliography: plain / apalike / amsalpha / ...
\bibliography{literature} %You need a file 'literature.bib' for this.
\end{document}
